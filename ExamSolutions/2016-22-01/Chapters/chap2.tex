\section{ Question 2:Volcanism and Planetary Interiors }\label{sec:q2}    

\textbf{(a)} The tectonic model of the plates in Earth gives a nice picture of the location of both Earthquakes and volcanoes. There exist between 10-15 plates in Earth, and the boundaries of the plates cluster both earthquakes and volcanoes. On the one hand, earthquakes mainly occur where two plates meet each other. There are separations, such as the ocean ridge in the middle of the Atlantic ocean, where the plates diverge. We can also see earthquakes at the ring of the pacific and at the separation of the Nasca plate under the South american plate. On the other hand, 2/3 known types of volcanoes can be explained with subduction of the plates: when the oceanic lithosphere goes into the continental plate, the oceanic sinks can melt as temperature rises, and percolate through the continental crust. Also, separation of two plates. Third type: hot spot volcanism (hawaii). 

\textbf{(b)} Radial structure of the four terrestial planets is similar: 1) iron nickel core, 2) mantle with silicates. In mercury, the mantle is really small and fluid (magnetic field). 3) Outer lithosphere which can be thicker (Mars, Mercury) or thinner (Venus, 20km). 4)Similar size (order of magnitude): 3500, 6500, 6000 and 2500 km. The cause is that they are formed in the same moment, after late heavy bombardment. 

\textbf{(c)} Earth tectonics are based on subduction. In Venus, there are plume or blob tectonics, due to the high surface temperatures, which causes plates to be too ductile to subduct, and the absence of water, which causes plates to be too stiff to subduct. The outcome of this internal dynamics is that the interior heats up due to current heat flow, so every few $10^8$ years a catastrophic resurfacing event ocurs.

\textbf{(d))} Cryovolcanism is cold volcanism: In planets where the temperature is really los, such as Saturns moon Triton, icy rocks can flow like lavas, and pressure release of trapped fluids can manifest itself as ejecting plumes up to 8 km above triton surface. In the tiny moon Enceladus, “tiger-stripe patterns” can be seen near its south pole. These patterns can be simply fractures in the overlying solid water ice layer due to pressurized liquid water causing plumes.